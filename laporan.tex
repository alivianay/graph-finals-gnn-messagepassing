\section{Dataset, Sampling, dan Subgraph}

Penelitian ini menggunakan dataset \textit{Cora}, yaitu dataset benchmark yang umum digunakan untuk tugas \textit{node classification} pada \textit{Graph Neural Network} (GNN). Dataset Cora merepresentasikan jaringan sitasi (\textit{citation network}), di mana setiap node merepresentasikan sebuah dokumen ilmiah dan setiap edge merepresentasikan hubungan sitasi antar dokumen. Tujuan dari tugas klasifikasi pada dataset ini adalah menentukan kategori topik dari setiap dokumen berdasarkan informasi fitur dan struktur graf.

Berdasarkan hasil pemuatan dataset menggunakan \textit{library} Torch-Geometric, dataset Cora memiliki sebanyak 2.708 node, 10.556 edge, dan 7 kelas yang merepresentasikan kategori dokumen. Setiap node memiliki 1.433 fitur yang direpresentasikan dalam bentuk vektor biner (\textit{bag-of-words}) yang bersifat \textit{sparse}. Representasi fitur ini menunjukkan keberadaan kata tertentu dalam dokumen, sehingga sebagian besar nilai fitur bernilai nol. Secara struktural, graf Cora bersifat tidak berarah (\textit{undirected graph}) dan menunjukkan kecenderungan \textit{homophily}, yaitu node-node yang saling terhubung cenderung memiliki label kelas yang sama.

Secara internal, Torch-Geometric menyimpan dataset dalam bentuk objek \texttt{Data}, yang terdiri dari beberapa komponen utama. Matriks fitur node direpresentasikan oleh variabel \texttt{x}, struktur konektivitas graf direpresentasikan oleh \texttt{edge\_index}, dan label kelas node direpresentasikan oleh \texttt{y}. Selain itu, dataset juga menyediakan \textit{mask} untuk data latih, validasi, dan uji, yang digunakan pada tahap pelatihan dan evaluasi model.

Untuk meningkatkan efisiensi komputasi dan menghindari pelatihan pada keseluruhan graf, penelitian ini menerapkan pendekatan \textit{node sampling}. Sebanyak 50 node dipilih secara acak sebagai \textit{node of interest} atau \textit{seed nodes}. Proses sampling dilakukan dengan menetapkan \textit{random seed} agar hasil eksperimen bersifat reprodusibel. Dari node-node terpilih tersebut, dilakukan pembentukan subgraph menggunakan pendekatan \textit{k-hop neighborhood} dengan nilai $k = 2$. Artinya, subgraph mencakup node-node yang berada hingga dua hop dari setiap \textit{seed node}. Pembentukan subgraph dilakukan menggunakan fungsi \texttt{k\_hop\_subgraph} dari Torch-Geometric, yang secara otomatis mengekstraksi node dan edge yang relevan serta melakukan \textit{relabeling} node agar indeksnya konsisten pada subgraph.

Hasil proses pembentukan subgraph menghasilkan graf dengan 842 node dan 3.224 edge. Subgraph ini merepresentasikan sebagian dari graf asli Cora, namun tetap mempertahankan struktur lokal yang penting untuk proses pembelajaran GNN. Data subgraph selanjutnya disusun kembali dalam objek \texttt{Data}, yang berisi fitur node, struktur edge, dan label kelas yang sesuai, dan digunakan sebagai input pada tahap pra-proses dan pelatihan GNN.
